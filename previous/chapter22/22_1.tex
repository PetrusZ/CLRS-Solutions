\subsection*{Problem 22-1 Classifying edges by breadth-first search}
\begin{enumerate}
	\item	\begin{enumerate}
			\item	对一个无向图进行BFS的时候,若有back edge或是forward edge,利用BFS的性质和无向图的性质,其在BFS Tree中都是tree edge
			\item	对于一条tree edge $(u, v)$,我们有$\attrib{v}{\pi} = u$,即利用BFS性质,可得$\attrib{v}{d} = \attrib{u}{d} + 1$
			\item	若边$(u, v)$是cross edge,则当我们访问$u$的时候,$v$必然已经进入队列,否则$(u, v)$为tree edge;由Lemma 22.3,我们有$\attrib{v}{d} \leq \attrib{u}{d} + 1$;由Corollary 22.4,我们有$\attrib{v}{d} \geq \attrib{u}{d}$,即$\attrib{v}{d} = \attrib{u}{d}$ or $\attrib{v}{d} = \attrib{u}{d} + 1$
		\end{enumerate}
	\item	\begin{enumerate}
			\item	在BFS过程中forward edge即为tree edge
			\item	与无向图相同
			\item	在BFS过程中,对于任意边$(u, v)$均满足$\attrib{v}{d} \leq \attrib{u}{d} + 1$
			\item	易得对于所有$v$,有$\attrib{v}{d} \geq 0$,因为$(u, v)$是back edge,即$v$是$u$的祖先,所以有$\attrib{v}{d} \leq \attrib{u}{d}$,即$0 \leq \attrib{v}{d} \leq \attrib{u}{d}$
		\end{enumerate}
\end{enumerate}

