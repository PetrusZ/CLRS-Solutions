\subsection*{Problem 8-7 The 0-1 sorting lemma and columnsort}
\begin{enumerate}
	\item	$A[p]$和$A[q]$一定都在错误的位置,且由定义有$A[p] < A[q]$,所以$B[p] = 0, B[q] = 1$
	\item	由于$A[p]$是最小的错误位置元素,则$q < p$,即$(p > q, A[p] < A[q]) \Rightarrow (p > q, B[p] = 0 < B[q] = 1)$;又因为Alogrithm X是一个oblivious compare-exchange algorithm,即Algorithm X排序A的操作指令和排序B时是一样的,所以Algorithm X不能正确的排序$B$序列
	\item	不管用什么算法来排序,排序后的结果都和一个oblivious compare-exchange algorithm执行之后的结果相同,所以我们可以认为columnsort是一个oblivious compare-exchange algorithm
	\item	因为完成step 1的时候,所有的1都在底部,所以在完成step 2, 3之后,右边的列的1的个数一定大于等于左边的列的1的个数,而dirty row一定是由一些0和一些1组成;即1都连续的排在右边,0都连续的排在左边,所以至多($<$)有$s$ rows是dirty rows
	\item	必然是从clean area of 0s开始,必然是在clean area of 1s结束;因为在step 3之后,至多有$s$ rows是dirty rows,所以中间的dirty area至多包含$s^2$个元素
	\item	step 5的目的是保证每一列自身有序,step 6的目的是将整个array首尾连接 \\
		因为$r / 2 \geq s^2$,又因为dirty area有至多$s^2$个元素,所以在step 5之后如果还未排好序,则一定是如下所示的这种情况
		\begin{center}
			\begin{tabular}{lll}
				0 & 0 & 0 \\
				0 & 0 & 0 \\
				0 & 0 & 0 \\
				0 & 0 & 1 \\
				0 & 1 & 1 \\
				0 & 1 & 1 \\
			\end{tabular}
		\end{center}
		即存在两个相邻的列,左边的列是以1结尾,而右边的列以0开头;所以在step 7, 8之后,所有的array中元素都按column-major order排好
	\item	如果$s$不能整除$r$,即对于every column在做变换的时候,至多产生2 dirty rows,即中间的0、1过度区域产生一个dirty row,最后一行产生一个dirty row;在step 3再排序之后,dirty cows的个数等于最多1的列和最少1的列相差1的个数,而对于1 column变换,即至多产生2 dirty rows,这2 rows中column之间的个数至多差1个,所以step 3之后dirty cows至多为$s$ \\
		$r \geq 2s^2$
	\item	如果$r \geq 2s^2$不成立,则可以通过填充无意义的元素(example. $\infty$)来使得$r \geq 2s^2$成立
\end{enumerate}

